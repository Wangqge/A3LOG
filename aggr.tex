\section{Preliminary}
\label{sec:async}

%A \emph{sequential} computation (as opposed to parallel computation) is composed of a sequence of operations on input. These operations can be classified into two categories, the aggregate operations and the non-aggregate operations. %We focus on numerical aggregation in this paper.
\subsection{Recursive Program with Aggregation}

\Paragraph{Aggregate Operation} An \emph{\textbf{aggregate operation}} is a function $g()$ with more than one input values, i.e., $g(Y)$ where $Y=\{y_1, y_2, \ldots, y_n\}$. For example, MAX, MIN, SUM, AVG, and COUNT are commonly used aggregate operations. A special but common case is the \emph{\textbf{group-by aggregation}}. A set of input key-value pairs (i.e., kv-pairs) are first grouped by key, and then an aggregate function is applied on each group to obtain a new set of kv-pairs. Given a set of kv-pairs $Y=\{\langle k_0,v1\rangle,\langle k_1,v2\rangle,\langle k_0,v3\rangle,\ldots\}$ with $n$ unique keys, the group-by aggregation $G(Y)$ applied on $Y$ will output $n$ new kv-pairs, each corresponding to the aggregation result of each group. On the contrary, a \emph{\textbf{non-aggregate operation}} is a function that takes only one value as input, i.e., $f(x)$. The result of non-aggregate operation $F(X)$ on a set of kv-pairs $X$ is the union of applying $f$ operation on each kv-pair.
%In other words, the group-by aggregation is composed of multiple aggregate operations applied on each group. 
%Given a set of kv-pairs $Y=\{Y_{k_0} \cup Y_{k_1} \cup ... \cup Y_{k_n}\}$, where $Y_{k_i}=\{\langle k_i,v1\rangle,\langle k_i,v2\rangle,\ldots\}$ denotes the set of kv-pairs that share the same key $k_i$ and there are totally $n$ unique keys. The group-by aggregation $G_k$ applied on $Y$ can be described as follows:
%\begin{equation}
%	\begin{aligned}
		 %Y&=\{Y_{k_0} \cup Y_{k_1} \cup ... \cup Y_{k_n}\}\notag\\	
%	 G_k(Y)&=\{g(Y_{k_0}) \cup g(Y_{k_1}) \cup \dots \cup g(Y_{k_n})\}
% \end{aligned}
%\end{equation}
%The result of the group-by aggregation is a new set of aggregated kv-pairs.
%Note that, there is no intersection between different subsets, i.e.$Y_i \cup Y_j = \emptyset $

%\begin{equation}
%	\begin{aligned}	
%	F_k(X)&=\{f(x_{k_0}) \cup f(x_{k_1}) \cup \dots \cup f(x_{k_m})\}
%	\end{aligned}
%\end{equation}
%Since the non-aggregate operation $f$ only takes one value as input, it is obvious that $F_k(X)$ has the distributive property, i.e., $F_k(X_1 \cup X_2)=F_k(X_1) \cup F_k(X_2)$.
%In essence, the aggregate operation implies a \emph{data fusion}, while the non-aggregate operation implies a \emph{data transformation}.

%Note that, the aggregate and non-aggregate operation should be judged according to the number of \emph{variable} inputs instead of constant inputs. For example, the join operation in relational algebra is in general an aggregation since it merges two data sets into one. It takes two inputs $A$ and $B$ and produces one output. However in distributed join implementation, the whole data set  $B$ can be small enough to be cached on all distributed workers, which implies that $B$ is a constant input for the distributed join operations. By partitioning and distributing $A$, the join operation is achieved by merging a part of $A$ and the locally cached whole $B$ on each worker. In such a case, the join operation is considered as a non-aggregate operation because each distributed join operation only takes a part of $A$ as a variable input which is different on different processor, where $B$ is the constant for all processors.

\Paragraph{Parallelization} Non-aggregate operation $f$ only takes one input and does not rely on any other variable, which makes $F$ operation have the distributive property, i.e., $F(X_1 \cup X_2)=F(X_1) \cup F(X_2)$. By partitioning and distributing the inputs to many workers, the non-aggregate operations can be embarrassingly parallel \emph{without communication} between workers. In contrast, the aggregate operation $g$ takes more than one values as input which may originate from other workers, so that the parallelization of group-by aggregation $G$ \emph{requires communication} between workers. If the inputs are produced by many other workers, the aggregate operations has to wait for all the inputs to be ready, which may incur significant coordination costs. Therefore, the efficiency of aggregation is the key to parallelization performance.


%If the inputs for all aggregate operations comes from same work, these aggregate operations can be parallelized without communication by carefully partitioning the inputs. However, this is usually impossible since the inputs probably result from the previous computations on other workers and cannot be particularly partitioned.In other words,the non-aggregate operations can be executed without waiting for all the inputs prepared, but the aggregate operations has to wait for all the inputs prepared when synchronously executed,which incur a lot of coordination costs.


\Paragraph{Recursive Program with Aggregation}  A \emph{recursive program with aggregation} is a sequence of interleaving aggregate and non-aggregate operations, which can be represented as follow.
\begin{equation}
\label{eq:recursive2}
\begin{aligned}
Y^{k}=F(X^k),\\
X^{k+1}=G(Y^k)
\end{aligned}
\end{equation}
This recursive program starts from $X^0$ and terminates when there there is no difference between $X^k$ and $X^{k-1}$ or when the difference is ``small'' enough. Let $(G\circ F)^n$ denote $n$ applications of $(G\circ F)$. The final result of the recursive program is $(G\circ F)^n(X^0)$. Since $X$ contains a set of kv-pairs, the computation of $F(X)$ can be parallelized with each processer working on a subset of $X$. The group-by aggregate computation $G(Y)$ can be parallelized with each processer performing one or several groups' aggregation. The aggregation for each group has to wait for its complete input kv-pairs set to be ready.

Recursively-defined aggregate functions are very common in graph algorithms, data mining and machine learning algorithms. We take a graph algorithm Single Source Shortest Path (SSSP) as an illustrative example.
%There is a special case of recursive datalog problem with aggregate operators which  can be represented  with \emph{recursive aggregations}. In this paper, we will focus on optimizing the performance of recursive aggregation programs.

%Let $Y$ denote a set of input variables of aggregate operation $\{Y_{k_0},Y_{k_1},\dots,Y_{k_m}\}$and $X$ denote the inputs of non-aggregate  operation.$\{x_{k_0},x_{k_1},\dots,x_{k_n}\}$%Define $F(X)$ as $\{f(x_0^k) \cup f(x_1^k) \cup \dots \cup f(x_n^k)\}$ that is performing non-aggregation function $f(\dot)$for each item in the set and then combine them. Similarly, We define $G(Y)$as $\{g(Y_0) \cup g(Y_1) \cup \dots \cup g(Y_n)\}$ in which $X_i$ is a subset of inputs X,note that there may be an intersection between any two subsets.
%We use $G()$ for $G_k()$ and $F()$ for $F_k()$ in the later text for brevity. Then a recursive program 
%It start from $X^0$ in which case the update order is exchanged and an aggregation is first applied \textcolor{red}{I don't understand your meaning this sentence}.


%\Paragraph{Example 1: Compuiting Paths in a DAG} This algorithm (denoted as \textbf{PATH}) counts the paths between all pairs of vertices in an acyclic graph. The number of paths between vertex $s$ and vertex $d$, $path(s,d)$, is initialized as $1$ for each edge $(s,d)$ and $0$ for others. The $f$ operation of the $k$th recursion for vertex pair $(s,d)$ takes $path^k(s,d)$ as input and outputs $path_{tmp}^k(s,d')$ if edge $(d,d')$ exists. The aggregation operation $g()$ with respect to each pair $(s,d')$ takes all $path_{tmp}^k(s,d')$ as inputs and computes $path^{k+1}(s,d')=\sum_{d'} path_{tmp}^k(s,d')+path^k(s,d')$ as the result. The computation terminates when the path numbers for all vertex pairs are not changed from previous recursion.
%{\color{green}
%\Paragraph{Example 1: SSSP} The single source shortest path (SSSP) computation is a recursive program that derives the shortest distance from a source node to all other nodes in a graph. The shortest distance is initialized as $+\infty$ for each node except for the source, which is initialized as 0. The $f()$ operation for a node $i$ takes a tuple $\langle i,d_i^k\rangle$ as input where $d_i^k$ is the shortest distance in the $k$th recursion, computes $f(d_i^k)=d_i^k+w_{i,j}=td_j^{k+1}$ for any outgoing neighbors $j$ (where $w_{i,j}$ is the distance from node $i$ to $j$), and outputs the tuples set $\{\langle j,td_j^{k+1}\rangle\}$ and its own $\langle i,d_i^k\rangle$. The aggregate operation $g()$ with respect to each node $j$ takes the input tuples $\{\langle j,td_j^{k+1}\rangle\}$, performs aggregation $g(td_j^{k+1})=min(\{td_j^{k+1}\},d_j^k)=d_j^{k+1}$, and outputs $\langle j,d_j^{k+1}\rangle$. It terminates when all nodes' shortest distances are not changed from previous recursion.
%}
\begin{comment}
\begin{verbatim}
Program 1. Single Source Shortest Path
\end{verbatim}
\vspace{-0.1in}
\small
\begin{lstlisting}
r1. sssp(X,$d$)$\leftarrow$ X=1,$d=0$.
r2. sssp(Y,min[$d$])$\leftarrow$ sssp(X,$d1$),edge(X,Y,$d2$),
$d=d1+d2$,sssp(Y,$d$).
\end{lstlisting}
\normalsize
\end{comment}

\Paragraph{Example 1: SSSP} The SSSP computation is a typical recursive program. The shortest distance is initialized as $+\infty$ for each node except for the source, which is initialized as 0. For each node $i$, it takes the shortest distance $\langle i, d^k\rangle$ as input and performs $f(\langle i, d^k\rangle)$, which 1) outputs $\{\langle j,d_{tmp}\rangle\}$ to node $i$'s outgoing neighbors $j$ where $d_{tmp}=d^k+w_j$ and 2) outputs $\langle i,d^k\rangle$ to itself. The aggregate operation $g$ with respect to each node $j$ takes the set of input tuples $\{\langle j,d_{tmp}\rangle\}$, performs aggregation $g(\{\langle j,d_{tmp}\rangle\})=\langle j, min(\{d_{tmp}\})\rangle$, and outputs tuple $\langle j,d^{k+1}\rangle$. It terminates when all nodes' shortest distances are not changed from previous recursion.

%\Paragraph{Example 2: PageRank} The PageRank computation is another typical recursive program for ranking the nodes in a graph. The ranking score is initialized as $r_i^0=1/|V|$ for each node $i$ where $|V|$ is the total number of nodes. The $f()$ operation for node $i$ takes a tuple $\langle i,r_i^k\rangle$ as input where $r_i^k$ is the ranking score in the $k$th recursion, computes $f(r_i^k)=0.85*r_i^k/d_i=tr_j^{k+1}$ for any outgoing neighbors $j$ (where 0.85 is the constant damping factor and $d_i$ is the out-degree of node $i$), and outputs the tuples set $\{\langle j,tr_j^{k+1}\rangle\}$. The aggregate operation $g()$ with respect to each node $j$ takes the input tuples $\{\langle j,tr_j^{k+1}\rangle\}$, performs aggregation $g(\{tr_j^{k+1}\})=\sum_j{tr_j^{k+1}}+0.15=r_j^{k+1}$ and outputs $\langle j,r_j^{k+1}\rangle$. It terminates when the difference between two continuous recursions' ranking scores is small enough.

\subsection{Datalog and Evaluation Techniques}

A Datalog program is a set of rules. A \emph{rule} \texttt{r} has the form $h\leftarrow b_1,\ldots,b_n$, where $h$ is the \emph{head} and $b_1,\ldots,b_n$ is the \emph{body}. The comma separating literals in a body is a logical conjunction (AND). $h$ and $b_i$ can be with the form $p_i(t_1,\ldots,t_j)$ where $p_i$ is a \emph{predicate} and $t_1,\ldots,t_j$ are terms which can be \emph{constants}, \emph{variables} or \emph{functions}. For example, the Datalog program for SSSP is as follows.
\begin{verbatim}
Program 1. Single Source Shortest Path
\end{verbatim}
\vspace{-0.1in}
\small
\begin{lstlisting}
r1. sssp(X,$d$)$\leftarrow$ X=1,$d=0$.
r2. sssp(Y,min[$d$])$\leftarrow$ sssp(X,$d1$),edge(X,Y,$d2$),
$d=d1+d2$;sssp(Y,$d$).
\end{lstlisting}
\normalsize
The first rule initializes the distance to the source node (with ID 1) as 0. The second rule is a recursive statement declaring that there is a path from source to node Y with length $d1+d2$, if there is a path from source to X with length $d1$ and an edge from X to Y of length $d2$. The shortest path from source to Y is simply the shortest of all the possible paths from source to Y, as expressed by the $\$min$ operation in the head.  

%The distance are computed in rule body by $d=d1+d2$ to all the outgoing neighboor(non aggregate operation), and then the aggregate function $\$min$ is applied to find minimal distance. 
Suppose the initial result set is $X^0$, the naive evaluation of Datalog program recursively applies $(G\circ F)$ operation on $X^0$ and finally obtains $(G\circ F)^k(X^0)$ after $k$ iterations. But naive evaluation is inefficient since it re-produces the existing results already contained in previous iterations, which incurs many redundant computations.

\textit{Semi-naive evaluation} is an essential optimization to the efficient execution of recursive Datalog programs. It avoids redundant computation by incrementally computing the results. Semi-naive evaluation can be expressed as follows.
\begin{equation}
\label{eq:seminaive}
\begin{aligned}
X^{i}&=G(X^{i-1} \cup F(\Delta X^{i-1})),\notag\\
  \Delta X^i&=X^i-X^{i-1}, 
\end{aligned}
\end{equation}
where `$-$' is \textcolor{red}{not set subtraction, there is also an explicit aggregation here, e.g., \{<k1,5>,<k2,3>\}-\{<k1,6>\}=\{<k1,5>,<k2,3>\}} and $\Delta X^0=X^0$. The final result is the aggregation of all the results obtained in each iteration, i.e., $R=G(\cup_{i=1}^{k}F(\Delta X^i)\cup X^0)$. Seo et al. \cite{Lam:2013:SDE:2510649.2511289} show that semi-naive evaluation can be applied to recursively-defined aggregation functions, if $G$ are \textit{meet} operations (associative, commutative, and idempotent, e.g., min and max) and that $F$ is monotonic under the partial order induced by the meet operations. Wang et al. \cite{Wang:2015:AFR:2824032.2824052} generalize the conditions by allowing $G$ to be \textit{bag-monotonic} aggregate operations, such as min, max, count and sum. It also requires $F$ to be monotonic with respect to the order defined by $G$.

%\textcolor{red}{Please add datalog background, and rewrite the blue part, introducing semi-naive evaluation by referring other datalog papers.}

%\textcolor{blue}{If evaluated with naive technology, The recursive rules will be repeatedly evaluating on the whole datasset untill terminate conditions are satisfied. Which incurs a lot of unnecessary computation. studies\cite{Seo:2013:DSD:2556549.2556572,Wang:2015:AFR:2824032.2824052,Lam:2013:SDE:2510649.2511289} showes that evaluating the origin recursive aggregation with naive-evaluation technology is very inefficient in datalog system. So seminaive-evaluation technology has been proposed\cite{Lam:2013:SDE:2510649.2511289},  the seminaive form is shown in equation \ref{eq:seminaive}, in which $-$ is set subtraction. This implies that only non convergence result will involved in the next step evaluation. Comparing with naive evaluation technology,semi-naive technology can puring much redudent computation. and acclerating the convergence. }





%Myria \cite{} further proposed $bag$-$monotonic$ conditions. An aggregate function $G_m$: $S\rightarrow V$ and a partial order $\preceq$ on $V$ in which $S$ is a tuple set and $V$ is the kv-pair set. $G_m$ is $bag$-$monotonic$ if for any $x,y\in S \wedge x\preceq y$, we have $G^m(x)\preceq G^m(y)$ .The application scope to more popular aggregate functions including $\$min$, $\$max$, $\$sum$ and $\$count$. These work shows that if aggregate operator is $bag$-$monotonic$ and non-aggregate operation $F$ is monotonic, the recursive program can be evaluated with semi-naive technology. Comparing with naive evaluation technology,semi-naive technology can puring much redudent computation. and acclerating the convergence. }